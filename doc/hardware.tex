% compile command line: `$ pdflatex hardware.tex`
\documentclass[a4paper,10pt]{article}

\usepackage[utf8]{inputenc}
\usepackage[siunitx]{circuitikz}
\usepackage[margin=1in]{geometry}
\usetikzlibrary{fit,arrows.meta,matrix,positioning}

\title{xr25\_diag: hardware interface}
\author{Javier Lopez-Gomez}
\date{October 19, 2016}

\begin{document}

\maketitle
\section{Introduction}
This document briefly documents the hardware required to interface the ECU
diagnostics port.
For other instructions, see \texttt{README.md} in the parent directory.

\section{Hardware}
The hardware is based on a FTDI FT232RL RS232$\leftrightarrow$USB converter;
off-the-shelve hardware can be used here, e.g. the Arduino Serial-to-USB mini.
Hardware drivers are most likely bundled ad part of your GNU/Linux distribution.
Thus, the device should appear as \texttt{/dev/ttyUSB0} or similar.
In addition to the FT232RL, the circuit in figure \ref{fig_xr25circuit} is required
to adapt the ECU voltage levels to the FT232RL inputs.
The \texttt{RXD} input is to be connected to the FT232RL RXD pin.
This schematic is provided for reference; other circuits can be found on the Internet.

\begin{figure}[!h]
  \ctikzset{bipoles/length=0.8cm}\centering
  \begin{circuitikz}
    % BC557
    \draw (0,0) node[pnp,xscale=-1] (pnp1) {} (pnp1) node[xshift=-20pt] {BC557}
    (0,3) node[vcc] {VCC} to [short] (0,2) node[circ] (circ1) {} to (pnp1.E)
    (pnp1.B) to [short] (1,0) node [circ] (circ2) {} to [R=390] (3,0) to
    % J1 and J2
      [D*,name=d1] (4,0) to [short,-o] (6,0) node[anchor=west] {J1}
      (d1.north) node[anchor=south] {1N4148}
    (6,-1) node[anchor=west]{J2} to [short,o-] ++(-1,0) |- (5,-3) node[vee]{VSS}
    % pnp1.B-pnp1.E resistor
    (circ2) to [R=390] ++(0,2) -| (circ1)
    % pnp1.C part
    (pnp1.C) to [leD*] (0,-2) node[circ] (circ3) {} to [R=1k8] (0,-3) node[vee]
      {VSS}
    % RXD
    (circ3) to [short,-o] ++ (-2,0) node[anchor=east] {RXD};
  \end{circuitikz}
  \caption{XR25 receiver schematic}\label{fig_xr25circuit}
\end{figure}

The \texttt{J1} and \texttt{J2} pins should be connected, respectively to pins $2$ and
$9$ on the Renault 12-pin diagnostics port, as shown in figure \ref{fig_connector}.
\begin{figure}[!h]
  \tikzset{connector/.style={inner sep=3pt,draw,
      column sep=3pt,row sep=3pt},
    pin/.style={font=\large\ttfamily,draw},pin label/.style={font=\large,
      color=black},
    connector key/.style={draw,minimum height=6pt,minimum width=14pt,
      node contents=,inner sep=0pt},
    ,>={Latex[length=4pt]}}\centering

  \begin{tikzpicture}[pin/.append style={minimum width=18pt,minimum height=14pt}
      ,connector key/.append style={minimum height=3pt},
      connector/.append style={
        after node path={node (__tmp) [inner sep=0pt,fit=(\tikzlastnode)] {}
          node[connector key,above=-0.1pt of __tmp.north,xshift=-21pt]
          node[connector key,above=-0.1pt of __tmp.north,xshift=21pt]
          node[connector key,below=0.1pt of __tmp.south]}}]
   \matrix (con) [connector,matrix of nodes,nodes={pin}] {
     1 & 2 & - \\
     4 & 5 & 6 \\
     7 & 8 & 9 \\
     10&11 &12 \\
   };
   \foreach \i/\j/\k [count=\c] in {60/con-1-2/GND (J2),
     11/con-1-3/Connector key,
     -11/con-3-3/DATA (J1)} {
     \node (pin\c) [pin label] at (\i:3cm) {\k};
     \draw[<-,pin label] (\j) -- (pin\c);
   };
  \end{tikzpicture}
  \caption{Renault 12-pin diagnostic connector (front view)}\label{fig_connector}
\end{figure}

\end{document}
